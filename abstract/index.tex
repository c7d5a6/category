\documentclass[a4paper]{article}
%\special{papersize=148mm,210mm}% it is A5 paper size, I got from

\usepackage{amssymb, amsthm, amsmath}
\usepackage{tikz-cd}

%\usepackage[russian, english]{babel}
\usepackage{fontspec}

\setmainfont{EB Garamond}

\newcommand {\cat}{%
	\mathbf%
}
\newcommand {\domain}[1] {%
	\mathrm{dom}(#1)%
}
\newcommand {\codomain}[1] {%
	\mathrm{cod}(#1)%
}
\newcommand {\idarrow}[1][] {%
	\mathbf{Id}_{#1}%
}

\theoremstyle{definition}
\newtheorem{defn}{Определение}[section]
\newtheorem{ex}{Exercise}

\title{Category Abstract}
\author{Mikita Kukavenka}
\date\today

\begin{document}
\maketitle
$\idarrow asdaasd$
$\idarrow{asd}asd$

\begin{defn}
	A \emph{category} $\cat{C}$ consists of
	\begin{itemize}
		\item a collection of objects: $A$, $B$, $C$, \ldots
		\item a collection of arrows: $f$, $g$, $h$, \ldots
		\item f o r each arrow $ f $ o b j e c t s $\domain{ f }$ and
		$\codomain{f}$ called the \emph{domain} and
		\emph{codomain} of $f$. If $\domain{f}=A$ and
		$\codomain{f}=B$ , we also write $f:A\to B$,
		\item given $f:A\to B$ and $g:B\to C$ , so that
		$\domain{g}=\codomain{f}$ , there is an arrow
		$ g\circ f:A\to C$ ,
		\item an arrow $\idarrow[A ] : A\to A$ for every
		object $A$ of $\cat{C}$ ,
	\end{itemize}
	such that
	\begin{description}
		\item [(Associative law)] for every $f :A\to B$ ,
		$g :B\to C$ and $h :C\to C$ we have
		\begin{equation}
			h\circ ( g\circ f )=(h\circ g )\circ f ,
		\end{equation}
		\item [(Unit laws)] for every $ f :A\to B$ we have
		\begin{equation}
			f \circ\idarrow[A] = f= \idarrow[B] \circ f .
		\end{equation}
	\end{description}
\end{defn}

\begin{equation}
	\begin{tikzcd}
		A \arrow [r, "f"]
		  \arrow [dr, swap, "g \circ f"]
		  &
		B \arrow [dr, "g \circ h"]
		  \arrow [d, swap , "g"]
		  \\
		  {}&
		C \arrow [r, swap, "h"]
		  &
		D
	\end{tikzcd}
\end{equation}

\begin{tikzcd}[cells={nodes={}}]
\arrow[loop left, distance=3em, start anchor={[yshift=-1ex]west}, end anchor={[yshift=1ex]west}]{}{\mathrm{id}_A} \arrow{r} A 
& B\times C\arrow[loop right, distance=3em, start anchor={[yshift=1ex]east}, end anchor={[yshift=-1ex]east}]{}{\mathrm{id}_{B\times C}} 
\end{tikzcd}


\begin{equation}
\begin{tikzcd}
	A \arrow [loop right, "asd"] \arrow [loop right, distance=5em, "asd22"] \arrow [loop right, distance=10em, "asd22asd"]
\end{tikzcd}
\end{equation}
\end{document}
